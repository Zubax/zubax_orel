%
% Copyright (c) 2017-2018  Zubax Robotics  <info@zubax.com>
%
% Distributed under BY-NC-ND (attribution required, non-commercial use only, no derivatives).
%

\documentclass{zubaxdoc}
\graphicspath{{document_templates/documentation_template_latex/}}

\title{Zubax Orel 20 Datasheet}

\hbadness=10000

\begin{document}
\frontmatter

\begin{titlepage}

\section*{Overview}

Zubax Orel 20 is an advanced sensorless BLDC propeller drive controller with doubly redundant CAN bus interface.
Zubax Orel 20 runs the Sapog\footnote{Refer to the Sapog Reference Manual for information about the firmware.}
firmware.

A reduced version Zubax Orel 21 is available for cost-sensitive applications.
Unlike Orel 20, it lacks the second CAN bus interface.

\section*{Features}

\begin{itemize}
    \item Excellent dynamic characteristics.
    \item Regenerative braking and active freewheeling.
    \item 350 W continuous power output at 10 g weight.
    \item Optional RPM control loop (RPM governor).
    \item Self diagnostics and health status reporting.
    \item Highly configurable.
    \item Low noise and low current ripple due to the low-\allowbreak{}ESR embedded filtering capacitors and
    high-\allowbreak{}frequency PWM.
    \item Supported interfaces:
    \begin{itemize}
        \item CAN (ISO 11898-2), with optional dual redundancy. Orel 21 does not support redundant CAN bus.
        \item UART.
        \item RCPWM (analog PWM interface widely used in robotics).
    \end{itemize}
    \item High quality assurance:
    \begin{itemize}
        \item Every manufactured unit undergoes a strict testing procedure.
        The testing log for each produced unit is available to the user via the website at
        \url{https://device.zubax.com/device_info}.
        \item Protection against unlicensed (counterfeit) production by means of a digital signature
        installed on every manufactured unit.
    \end{itemize}
    \item Open source firmware -- Sapog (3-clause BSD license).
\end{itemize}

\BeginRightColumn
\section*{Applications}

\begin{itemize}
    \item Propeller drives for light unmanned aerial vehicles.
    \item Pump and propeller drives for un\-man\-ned wa\-ter\-craft.
\end{itemize}

\centering
\includegraphics[width=0.45\textwidth]{image}
\includegraphics[width=0.45\textwidth]{image2}
\includegraphics[width=0.45\textwidth]{bottom}

\end{titlepage}

\tableofcontents
\BeginRightColumn
\listoffigures
\listoftables

\mainmatter

\chapter{Overview}

Zubax Orel 20 is an advanced controller of sensorless BLDC motors designed for use in
cost sensitive applications.
Its primary application domains include propulsion systems of electric unmanned aircraft
and watercraft.

Zubax Orel 20 runs Sapog - an open source multiplatform BLDC controller firmware
developed by Zubax Robotics.
Please refer to the Sapog Reference Manual\footnote{Available from the Zubax Robotics website.}
for its usage information.
This datasheet is focused only on the hardware aspect of the product.

A reduced model Orel 21 is available for cost-sensitive applications, which lacks the second CAN bus interface.

\section{System integration}

Zubax Orel 20 is a single supply device, which means that the device does
not expose any power supply inputs except for the high power supply.
The 5 V rails of the CAN interfaces are not used by the device; rather,
they connect the 5 V rails of their respective CAN connector pairs directly together.

Application-specific integration documentation is available on the Zubax Knowledge Base at
\mbox{\url{https://kb.zubax.com}}.

\begin{figure}[hb]
    \centering
	\includegraphics[width=\textwidth]{can_daisy_chain}
	\caption{Connection of CAN nodes in non-redundant and doubly redundant CAN bus configurations.
	\label{can_daisy_chain_non_redundant}}
\end{figure}

\section{Quality assurance}

Every manufactured Zubax Orel 20 undergoes an automated testing procedure that validates that
the device is functioning as designed.
The test log for every manufactured device is available on the web at
\url{https://device.zubax.com/device_info}.
This feature can be used to facilitate traceability of purchased devices and
provide additional safety assurances.

Every manufactured device has a strong digital signature stored in its non-volatile memory
which proves the origins of the product and eliminates the risk of sourcing unlicensed or
counterfeit hardware.
This signature is referred to as Certificate of Authenticity (CoA).
Please refer to the section \ref{sec:certificate_of_authenticity} and visit the
\href{https://kb.zubax.com}{Zubax Knowledge Base} to learn more about
the certificate of authenticity and how it can be used to trace the origins of your hardware.

\section{Accessories}

Zubax Orel 20 can be used with the following accessories:

\begin{itemize}
    \item Enclosure (suitable for 3D printing). See section \ref{sec:enclosure}.
    \item UAVCAN Micro patch cable.
    \item UAVCAN Micro adapter cables.
    \item UAVCAN Micro termination plug.
    \item Cables compatible with the Dronecode Autopilot Connector Standard.
    \item Zubax Dronecode Debug Probe.
\end{itemize}

Please contact your supplier for ordering information.

\subsection{Enclosure}\label{sec:enclosure}

Zubax Orel 20 is intended for integration into the end system in the form of the bare PCB,
as this facilitates better heat dissipation, lower weight, and tighter arrangement of components
in the end device, all of which are desirable properties in the targeted application domains.

Shall it be desired to provide additional mechanical protection for the device,
the plastic enclosure pictured on the figure \ref{enclosure} can be used.
Please contact your supplier for the ordering information;
alternatively, visit \url{https://github.com/Zubax/zubax_orel} to download
the 3D printable enclosure models suitable for in-house manufacturing.

\begin{figure}[hb]
	\centering
	\includegraphics[width=0.45\textwidth]{enclosure-open}
	\includegraphics[width=0.45\textwidth]{enclosure-closed}
	\caption{Plastic enclosure.\label{enclosure}}
\end{figure}

\chapter{Characteristics}

\section{Absolute maximum ratings}

Stresses that exceed the limits specified in this section may cause permanent damage to the device.
Proper operation of the device within the limits specified in this section is not implied.

\begin{ZubaxSimpleTable}{Absolute maximum ratings}{|c X|c c|c|}
    Symbol            & Parameter                & Min  & Max & Unit \\
	$V_\text{inv}$    & Supply voltage           & -0.3 & 19  & V \\
	$T_\text{oper}$   & Operating temperature    & -50  & 125 & \degree{}C \\
	                  & UART/RCPWM input voltage & -0.3 & 7   & V\\
	                  & CAN H/L input voltage    & -4   & 16  & V\\
\end{ZubaxSimpleTable}

\section{Environmental conditions}

\begin{ZubaxSimpleTable}{Environmental conditions}{|c X|l|c c|c|}
    Symbol & Parameter & Note & Min & Max & Unit \\
	$T_\text{oper}$ & Operating temperature &                            & -40 & 105 & \degree{}C \\
	$T_\text{stor}$ & Storage temperature   &                            & -40 & 50  & \degree{}C \\
	$\phi_\text{oper}$ & Operating humidity & Condensation not permitted & 0   & 100 & \%RH\\
	$h_\text{oper}$ & Operating altitude    & Above mean sea level (MSL) &     & 10  & km\\
\end{ZubaxSimpleTable}

\section{Reliability}

Please contact Zubax Robotics for additional reliability and safety information.

\begin{ZubaxSimpleTable}{Reliability}{|c X|c|c|}
    Symbol & Parameter & Typ & Unit \\
	MTTF   & Mean time to failure & 100000 & hours \\
\end{ZubaxSimpleTable}

\section{Power characteristics}

\begin{ZubaxTableWrapper}{Power characteristics}
    \begin{ZubaxWrappedTable}{|c X|c c c|c|}
        Symbol & Parameter & Min & Typ & Max & Unit \\
    	$P$                 & Continuous power                    &      &      & 350  & W \\
    	$P_\text{peak}$     & Peak power                          &      &      & 1000 & W \\
    	$I_\text{inv}$      & Continuous DC current               &      &      & 20   & A \\
    	$I_\text{inv-peak}$ & Peak DC current                     &      &      & 55   & A \\
    	$I_\text{idle}$     & Idle current consumption            &      & 50   &      & mA \\
        $V_\text{inv}$      & Supply voltage\tnote{a}             & 9    & 14.8 & 18.5 & V \\
        $V_\text{TVS}$      & TVS\tnote{b}\space{} circuit
                              activation voltage                  & 19   &      & 21   & V \\
        $P_\text{TVS}$      & TVS circuit maximum power
                              dissipation                         &      & 1000 &      & W \\
        $E_\text{TVS}$      & TVS circuit energy absorption
                              capability\tnote{c}                 &      & 5    &      & J \\
    	$\theta_\text{JA}$  & Junction-to-air thermal resistance  &      & 50   &      & K/W \\
        $R_\text{DS-on}$    & FET drain-source on-state resistance&      & 1.6  & 1.9  & $\text{m}\Omega$ \\
        $R_\text{phase}$    & Cumulative resistance of the
                              inverter per phase                  &      & 5    &      & $\text{m}\Omega$ \\
                            & Inverter temperature measurement
                              error                               & -6   &      & +6   & \degree{}C \\
                            & Inverter temperature measurement
                              range                               & -55  &      & 125  & \degree{}C \\
    \end{ZubaxWrappedTable}
    \begin{tablenotes}
        \item [a] Suitable battery packs per chemistry type:\\
        \begin{ZubaxCompactTable}{|c c c|}
    	    Chemistry         & Nominal cell voltage [V] & Cells in series [S]\\
    	    $\text{LiCoO}_2$  & 3.7                      & \numrange{3}{4}\\
    	    $\text{LiFePO}_4$ & 3.3                      & \numrange{4}{5}\\
    	    $\text{NiCd}$, $\text{NiMH}$ & 1.2           & \numrange{10}{12}\\
    	    $\text{Pb}$       & 2.0                      & \numrange{6}{8}\\
        \end{ZubaxCompactTable}

        \item [b] Transient voltage suppression.

        \item [c] At $T = 25\text{\degree{}C}$, non-repetitive.
    \end{tablenotes}
\end{ZubaxTableWrapper}

\subsection{Regenerative braking}

During regenerative braking, the device performs energy transfer from the motor to the power supply network.
If the self resistance of the power supply network is not sufficiently low,
the regenerative energy transfer may lead to an increase of the supply voltage beyond
the safe operating limits.
This event will trigger activation of the transient voltage suppression (TVS) circuit,
which will absorb some of the excessive energy.
If the amount of recovered energy exceeds the absorption capabilities of the power supply
network and the TVS circuit, the device may incur a fatal damage.

Generally, batteries are capable of absorbing the energy recovered during braking without issues.
Problems may arise if the device is powered from a source that does not permit high reverse currents,
such as laboratory power supplies.
In that case it is advised to install additional buffer capacitors to act as an energy storage
during braking.

\subsection{Power connectors}

Zubax Orel 20 is equipped with bullet 3.5 mm power connectors.
Input power is provided via male connectors soldered on 100 mm long 16 AWG wires.
The motor phases are connected via female connectors soldered on 100 mm long 16 AWG wires.
Additional connection options are available upon request.

\section{Communication interfaces}

\subsection{CAN bus}

The device is equipped with a doubly redundant ISO 11898-2 CAN 2.0A/B interface.
Each CAN interface has two standard UAVCAN Micro connectors\footnote{Refer to
\url{http://uavcan.org} for more information on UAVCAN.}
joined in parallel.
The power rails of the connector pairs are not connected to the device's internal circuitry,
since Zubax Orel 20 does not consume or provide power to the CAN bus.

The device does not terminate the CAN bus internally.

CAN2 (the secondary CAN bus interface) can only be used in configurations with redundant CAN bus.
If the bus is not redundant, only CAN1 (the primary CAN bus interface) can be used.
Connectors of the unused CAN bus interfaces should be left empty.

The reduced model Zubax Orel 21 does not have the second CAN bus interface;
only CAN1 (the primary CAN bus interface) is available there.

\begin{ZubaxSimpleTable}{CAN bus connectors pinout}{|c X X X[3]|}
	Pin no. & Type         & Name      & Comment \\
	1       & Power        & PWR       & Not connected to the device's circuits internally.\\
	2       & Input/Output & CAN H     & \\
	3       & Input/Output & CAN L     & \\
	4       & Ground       & GND       & \\
\end{ZubaxSimpleTable}

\begin{ZubaxTableWrapper}{Characteristics of CAN bus interfaces}
	\begin{ZubaxWrappedTable}{|c X|c c c|c|}
		Symbol  & Parameter                                 & Min  & Typ  & Max  & Unit \\
		        & Bit rate                                  & 20   &      & 1000 & Kbps \\
		        & Positive-going input threshold voltage    &      & 750  & 900  & mV \\
		        & Negative-going input threshold voltage    & 500  & 600  &      & mV \\
		        & Differential output voltage, dominant     & 1.5  & 2.0  & 3.0  & V \\
		        & Differential output voltage, recessive    & -120 & 0    & 12   & mV \\
		        & Bus power rail\tnote{a}\space{} voltage   & -10  &      & 10   & V \\
		        & Inter-connector current\tnote{a}          & -1 &  & 1    & A \\
		        & Connector resistance during device lifetime &    & 30   & 50   & $\text{m}\Omega$ \\
	\end{ZubaxWrappedTable}
	\begin{tablenotes}
	    \item [a] The limit is imposed by the PCB.
	\end{tablenotes}
\end{ZubaxTableWrapper}

\subsection{Dronecode debug port}

The device features a Dronecode debug port available via the standard
Dronecode Mini debug connector (DCD-M)\footnote{Refer to the DroneCode documentation
for more information on standard connectors and communication interfaces.}.
The Dronecode debug port provides access to the device's CLI\footnote{Command line interface.}
via UART, and to the RCPWM input which is shared with the UART RX line.

UART and RCPWM \emph{must not be used simultaneously}.
If RCPWM is activated, it is \emph{prohibited} to connect UART,
as that may cause unpredictable behavior of the RCPWM interface.

\begin{ZubaxSimpleTable}{Dronecode Mini debug connector pinout}{|c X X X|}
	Pin no. & Type         & Name                & Comment \\
	1       & Power        & TPWR                & 3.3 V power output \\
	2       & Output       & UART TX             & \\
	3       & Input        & UART RX \& RCPWM RX & Pulled down with a resistor\\
	4       & Input/Output & SWDIO               & Not for production use \\
	5       & Input        & SWDCLK              & Not for production use \\
	6       & Ground       & GND                 & \\
\end{ZubaxSimpleTable}

\begin{ZubaxSimpleTable}{Dronecode debug port characteristics}{|c X|c c c|c|}
	Symbol  & Parameter                                 & Min  & Typ  & Max  & Unit \\
			& Low-level input voltage                   & -0.3 & 0    & 1.6  & V\\
			& High-level input voltage                  & 2.1  & 3.3  & 5.5  & V\\
			& Low-level output voltage                  & 0    & 0    & 0.5  & V\\
			& High-level output voltage                 & 2.8  & 3.3  & 3.4  & V\\
			& Source/sink current via data pins         &      &      & 10   & mA\\
			& UART RX / RCPWM RX pull down resistance   & 15   & 20   & 25   & $\text{k}\Omega$\\
	$V_\text{DCDP}$ & Power rail output voltage         & 3.2  & 3.3  & 3.4  & V\\
	$I_\text{DCDP}$ & Power rail load capability        &      &      & 3    & mA\\
	        & Connector resistance during device lifetime &    & 20   & 40   & $\text{m}\Omega$\\
\end{ZubaxSimpleTable}

\subsection{RCPWM input}

The RCPWM interface has a dedicated connection point near the edge of the PCB,
suitable for soldering wires directly to it.
This connection point is connected directly to the UART RX / RCPWM RX pin,
and so it does not constitute an independent interface.
Same conditions and limitations apply.

\section{Indication}

Zubax Orel 20 is equipped with a single RGB LED indicator for purposes of status indication.
The LED is located on the bottom side near the edge of the PCB.

\section{Mechanical characteristics}

The drawing \ref{drawing} documents the basic mechanical characteristics of Zubax Orel 20,
such as the placement of connectors and mounting holes.

Both connectors of the primary CAN interface are located on the top side of the board.
They are explicitly marked as \verb|CAN1| on the PCB silkscreen.
Connectors of the secondary CAN interface are located on the bottom side of the board,
and marked \verb|CAN2|.

For reference, the red (positive) power supply wire is connected to the top side of the board.

\begin{figure}[!hbt]
	\centerline{\includegraphics[width=1.1\textwidth]{drawing}}
	\caption{PCB drawing.\label{drawing}}
\end{figure}

\begin{ZubaxSimpleTable}{Mechanical characteristics}{|c X|l|c|c|}
    Symbol & Parameter & Note                          & Typ & Unit \\
	$m$    & Mass      & Power connectors not included & 10  & g \\
\end{ZubaxSimpleTable}

\chapter{Device identification}

\section{Hardware version number}

Zubax Orel 20 reports to the Sapog firmware the following hardware version number
(in the form major.minor): 1.1.

The hardware version number can be used to determine whether the Sapog firmware
is running on a Zubax Orel 20 or not.

\section{Certificate of authenticity}\label{sec:certificate_of_authenticity}

The Sapog firmware can report the certificate of authenticity (CoA),
if one is found installed on the hardware that the firmware is running on.
The details covering how to request the CoA from Sapog fall outside of the scope of this document;
the reader is advised to refer to the Sapog Reference Manual for explanations.

Every manufactured instance of Zubax Orel 20 contains a valid CoA, which is an RSA-1024 digital signature
computed over the following ASCII plaintext string hashed through SHA-512:

\verb|<UID>io.px4.sapog|

Where \verb|<UID>| is a placeholder for the 128-bit unique ID represented in raw bytes.
It can be seen that the overall length of the plaintext is always 28 bytes.

For example, consider an instance of Zubax Orel 20 which has the following unique ID:

\verb|33ffd505474735362934204300000000|

The plaintext would be the following (where \verb|\x| is a hexadecimal escape sequence):

\verb|\x33\xff\xd5\x05\x47\x47\x35\x36\x29\x34\x20\x43\x00\x00\x00\x00io.px4.sapog|

The valid signature (CoA) for the above plaintext would be the following:

\begin{minipage}{0.7\textwidth}
\begin{verbatim}
c1fbfaf5ab835be10c5c5db94b304f22beed4d58bcaf9cbc83a539f588ae6b32
ab570c0e82e1916c186e447390560df299f6085387ab08cc84d79542ca619c73
600e3e6b52ad3db65616f78d7eb845c159d72014c79daf5474ec3fb1499ed6c2
32fc94bb26481c8acdf5a3e0022daf17e80378a6137776222b65d59454066cbd
\end{verbatim}
\end{minipage}

The signature (CoA) of the plaintext can be verified against the following RSA-1024 public key
using the SHA-512 hash function:

\begin{minipage}{0.7\textwidth}
\begin{verbatim}
-----BEGIN PUBLIC KEY-----
MIGfMA0GCSqGSIb3DQEBAQUAA4GNADCBiQKBgQDLrAYWFBmjnYnDaktbSBtpdoqG
Vey7unzbVe8db+JF0i+kQfW3hT1/UEJo7hIImpSQhB5/AtNtQ1kKF6r3VmhdjqCD
naoZfTnnybMs4J+JNSMZheaVFy5lmmpzzi4a3eEd7g26Qid1sWcfamiGRIqynia6
e2YOkFIAloBGphQjxQIDAQAB
-----END PUBLIC KEY-----
\end{verbatim}
\end{minipage}

\end{document}
